\begin{apendicesenv}
      
    \partapendices

    %Corrigir erro da numeração dos apendices
    \setcounter{chapter}{0}
    \renewcommand{\thechapter}{\Alph{chapter}}%
    
    \chapter{Homenique Vieira Martins}
    C é uma linguagem muito incrível, nos seus quase 50 anos, 
    suas estrutura e vasta documentação mostra a maturidade que
     os desenvolvedores foram criando ao longos dos anos no desenvolvimento
    da linguagem e começa a ficar mais claro como uma linguagem tão velha possui
     atualizações até nos dias de hoje, continua sendo muito utilizada e porque é
      um marco tão grande para a computação. 

    Além disso,estudando para esse trabalho, é notável a detalhada documentação de C,
     onde no final você aprende ainda mais sobre C.

     \begin{figure}[ht]
        \caption{\href{https://github.com/id-Software/DOOM}{Doom Engine}}
        \includegraphics[width =\textwidth]{doom-hero.jpg}
        \textbf{Fonte:} \url{https://images.ctfassets.net/rporu91m20dc/7bt6yTABm9pfXTyfbjrqb2/a78a456901b75994349247c260d8b48d/doom--1993--hero-img?q=70&fm=webp} 
      \end{figure}

    \nocite{Doom}
    \newpage


       
    \newpage
    \chapter{Rafael Amauri Diniz Augusto}

    A linguagem C é um marco gigantesco na história da computação.
    Sem ela, muitas das tecnologias que temos hoje em dia não existiriam 
    ou seriam diferentes ao ponto de serem irreconhecíveis. A presença de
    C no mercado é gigantesca até 49 anos depois da sua criação, e isso se
    deve à sua portabilidade e eficiência. C é uma linguagem extremamente 
    próxima de machine-language ao mesmo tempo que tem suporte universal 
    para os mais diversos tipos de arquiteturas, e tudo isso faz ela ser
    uma linguagem extremamente rápida e portátil. 

    \begin{figure}[ht]
        \caption{\href{https://github.com/torvalds/linux}{Github da Kernel de linux }}
        \includegraphics[width =\textwidth]{linux.png}
        \textbf{Fonte:} \url{https://digitalinnovation.one/artigos/conheca-a-historia-do-linux}
        \nocite{KernelLinux}
    \end{figure}
    
  
    \newpage
    


\end{apendicesenv}