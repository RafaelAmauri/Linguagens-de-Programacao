\chapter{Paradigma}


Podemos definir paradigma da programação como a forma que uma linguagem é construída,
 quais são suas regras, suas estruturas e o comportamento esperado. Neste trabalho,
  iremos discutir sobre o principal paradigma da linguagem C, neste caso, o paradigma Imperativo.

O paradigma imperativo é o mais antigo da história da computação, sendo baseado na estrutura de máquina 
do cientista Húngaro-americano John Von Neumann. O funcionamento de linguagens imperativas está fortemente
 ligado ao funcionamento da própria máquina, e por isso linguagens que utilizam essa estratégia tendem a ser mais rápidas, 
 pois o código é uma Transliteração das operações de máquina, ou seja as operações são uma função direta das operações existentes no hardware.  

Outros pontos notáveis incluem a execução do código de forma sequencial, com o uso de funções sendo apenas uma ferramenta adicional
e não necessariamente obrigatória, salvo o uso da função \emph{main()}.

Linguagens imperativas também são sempre baseadas em comandos, que são \emph{keywords} 
reservadas da própria linguagem que efetuam operações únicas e servem como instruções para o compilador ou interpretador.
Por fim, o paradigma imperativo também consiste no armazenamento de dados alterando as próprias células da memória principal.


Uma linguagem baseada no paradigma imperativo possui como características:

\begin{itemize}
    \item As variáveis  modelam as células de memória
    \item Os comandos de atribuição, que são baseados nas operações de
    transferência dos dados e instruções.
    \item A execução sequencial de instruções.
    \item A forma iterativa de repetição, que é o método mais eficiente desta arquitetura.

\end{itemize}
\nocite{paradigmaimperativo}
\nocite{Imperative_programming}
\nocite{Imperativeprogramming}
%Fim da pagina de paradigma
\newpage