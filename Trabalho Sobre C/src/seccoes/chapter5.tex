\chapter{Linguagens Relacionadas}


\begin{enumerate}
    \item Java
  
    Similaridades: 
        \begin{itemize}
            \item Sintaxe.
            \item Ambas são linguagens imperativas.
            \item Permite acesso a ponteiros.
            \item Ambas são grandes nomes no mercado de IoT.
        \end{itemize}
    Diferenças:
    \begin{itemize}
        \item Java compila para bytecode que será usado na JVM. C compila para machine code.
        \item C não possui classes, Java possui.
        \item C não possui um garbage collector, enquanto Java possui.
        \item Por causa da JVM, a portabilidade de Java é bem maior
    \end{itemize}
    Exemplo de código: 
    \lstinputlisting[language=java]{code/HelloWorld.java}


    \item Python
        
    Similaridades: 
        \begin{itemize}
            \item Ambas são linguagens com múltiplos paradigmas.
            \item Semântica.
            \item Possui suporte a tipos dinâmicos e primitivos.
        \end{itemize}
    Diferenças:
        \begin{itemize}
            \item Python é uma linguagem interpretada.
            \item Python não requer uma função main, enquanto C sim.
            \item Python não tem ponteiros.
            \item Python tem um garbage collector.
        \end{itemize}
    Exemplo de código: 
    \lstinputlisting[language=Python]{code/HelloWorld.py}
        
    \item C++
        
    Similaridades: 
        \begin{itemize}
            \item Ambas são multi-paradigma.
            \item Ambas permitem manipulação de ponteiros.
            \item Ambas não possui garbage collector.
            \item Ambas são linguagens compiladas.
        \end{itemize}
    Diferenças:
        \begin{itemize}
            \item C++ suporta sobrecarga de funções.
            \item C++ possui alocação dinâmica na memória.
            \item Por ser pesadamente voltada a POO, C++ possui suporte a polimorfismo, herança e encapsulamento.
            \item C++ possui suporte direto para exceções.
        \end{itemize}
    Exemplo de código: 
    
    \lstinputlisting[language=c++]{code/HelloWorld.c++}

    \item Haskell
        
    Similaridades: 
        \begin{itemize}
            \item Haskell requer uma função main.
            \item Ambos aceitam funções como parâmetros de outras funções.
            \item Ambas permitem manipulação da memória por ponteiros.
            \item Possui suporte a tipos dinâmicos e primitivos.
        \end{itemize}
    Diferenças:
        \begin{itemize}
            \item Haskell é uma linguagem interpretada.
            \item Haskell é uma linguagem funcional.
            \item Todos dados em Haskell são imutáveis.
            \item Todas funções em Haskell são funções puras.
        \end{itemize}
    Exemplo de código: 

    \lstinputlisting[language=Haskell]{code/HelloWorld.hs}    
\end{enumerate}

\newpage