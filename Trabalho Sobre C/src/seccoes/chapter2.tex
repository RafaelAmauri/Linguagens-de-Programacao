\chapter{História e Cronologia}

A versão original de C é conhecida como "K\&R C" por programadores.
Como a popularidade de C cresceu rapidamente nas décadas de 1970 e 1980,
foi criado um comitê para padronizar a linguagem e prover especificações 
consistentes. Essa versão ficou conhecida como ANSI C,
e é a versão mais utilizada da linguagem hoje em dia.

A partir de 1983, quando o padrão ANSI C foi criado, foram adicionadas 
diversas revisões da linguagem, que adicionam novas funcionalidades e recursos a cada revisão. A última 

\begin{itemize}
    \item Em C99 
    foram introduzidos suportes para funções inline, suporte a vários tipos novos de dados (como long long int)
    e suporte para comentários de uma única linha, começando com \(\backslash\backslash\) 

    \item Em C11 foram feitas várias novas adições à linguagem, como melhor suporte ao Unicode, 
    multi-threading, tipagem genérica e melhor compatibilidade com C++.

    \item Em C17 curiosamente não foram introduzidas funcionalidades novas, apenas correções de bugs e outras correções técnicas.

\end{itemize}


\nocite{languageC}
\nocite{DevelopmentOfTheCLanguage}
\nocite{wikipediaC}

%Fim da Pagina de Histórico 
\newpage