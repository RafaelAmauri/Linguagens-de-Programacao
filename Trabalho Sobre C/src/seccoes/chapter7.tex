\chapter{Projetos reconhecidos que usam C}

Sistemas Operacionais favorecem muito o uso da linguagem C para seu desenvolvimento por causa da proximidade da 
linguagem com o hardware, o suporte a múltiplas arquiteturas (como x86, x86\_64, arm, arm64 e RISC-V),

\begin{itemize}
    \item Unix, Linux Kernel, MacOS, Windows NT Kernel, FreeBSD
    Sistemas Operacionais favorecem muito o uso da linguagem C 
    para seu desenvolvimento por causa da proximidade da linguagem com o hardware, 
    o suporte a múltiplas arquiteturas (como x86, x86\_64, arm, arm64 e RISC-V), a altíssima 
    qualidade e confiança nos compiladores de C (GCC e Clang principalmente), e alto nível de 
    controle sobre o assembly code que vai ser gerado a partir do código-fonte. Abaixo seguem 
    exemplos de alguns sistemas operacionais que são desenvolvidos com a linguagem C.
\nocite{Unix}
\nocite{KernelLinux}
\nocite{macOS}

    \item AndroidOS, iOS
    importam muito do desenvolvimento que foi feito no Linux e no MacOS, respectivamente.
    A portabilidade de código em C é um dos motivos para essa escolha, já que compilar para outra arquitetura
    em C muitas vezes envolve apenas adicionar uma flag de compilação.
\nocite{Android}
    \item Arduino
    O suporte aos mais diversos tipos de dispositivos é essencial para trabalhar com Arduino, e o suporte de C a
    diferentes dispositivos provê justamente essa questão de portabilidade e suporte. O hardware dentro do Arduino
    também é comparativamente fraco quando comparado a um PC ou um celular, então utilizar uma linguagem compilada
    que minimiza o overhead de runtime e consumo de memória é um ponto importante.
    \nocite{arduino}

    \item PostgreSQL, MySQL, Oracle Database
    Fazer consultas em bancos de dados é uma operação extremamente custosa, e websites e aplicações muitas vezes precisam
    dos resultados dessas operações para mostrar algum elemento, então velocidade de tempo de execução e minimizar o gasto
    de memória são os fatores mais importantes na hora de fazer esses programas. Por conta do controle excelente sobre memória
    que a linguagem C oferece e pela proximidade com o hardware, C é uma excelente opção para o desenvolvimento de gerenciadores de bancos de dados.
    
    \item Blender
    Blender é uma aplicação que também precisa de velocidade e eficiência, já que ela lida com enormes quantidades de dados e fazem muitos cálculos por 
    segundo. Quanto mais eficientes elas. 
    \nocite{Blender}

    \item Id Tech 1, mais conhecida como Doom Engine
    
    Criada por John Carmack, a Doom Engine foi uma engine revolucionária pois 
    criava uma ilusão de 3D nunca vista antes, além de estabelecer novos paradigmas
    para o desenvolvimento de jogos. Usando a eficiência de C, fórmulas simplificadas 
    reduzidos muitos o custos computacionais e criando uma engine leve e eficiente.   
    \nocite{RETROCOMPATIBILIDADE}
    \nocite{Id_Tech} 

\end{itemize}

\newpage